\documentclass[12pt, a4paper]{article}

\usepackage[hmargin=2.5cm, vmargin=2cm]{geometry}
\usepackage{amsthm, amssymb, mathtools, yhmath, graphicx}
\usepackage{fontspec}
\usepackage{type1cm, titlesec, titling, fancyhdr, tabularx}
\usepackage{color, float, hhline}

\usepackage{algorithmicx, algorithm, algpseudocode}
\usepackage[CheckSingle, CJKmath]{xeCJK}
\usepackage{CJKulem}
\usepackage{enumitem}
\setenumerate{itemsep=0pt}
\usepackage[abbreviations, binary-units]{siunitx}
\DeclareSIUnit\mybyte{Byte}
\usepackage[usenames]{xcolor}
\usepackage{tikz}
\usepackage{circuitikz}
\usepackage{hyperref}
\usepackage{minted}
\usepackage{lipsum}
\BeforeBeginEnvironment{minted}{\vspace*{-5mm}}
\AfterEndEnvironment{minted}{\vspace*{-5mm}}

\usepackage{titling}
\setlength{\droptitle}{-3em}

\usetikzlibrary{matrix}
\usetikzlibrary{decorations.markings}
\usetikzlibrary{decorations.pathmorphing}

\usepackage[backend=biber, style=verbose]{biblatex}
\bibliography{final.bib}


\setmainfont{Linux Libertine O}
\setmonofont{Source Code Pro}
\setCJKmainfont{Source Han Sans TC}
\setCJKsansfont{Source Han Sans TC}
\setCJKmonofont{Source Han Sans TC}

\def\codesize{\fontsize{10}{15}\selectfont}
\def\normalsize{\fontsize{12}{18}\selectfont}
\def\large{\fontsize{14}{21}\selectfont}
\def\Large{\fontsize{16}{24}\selectfont}
\def\LARGE{\fontsize{18}{27}\selectfont}
\def\huge{\fontsize{20}{30}\selectfont}


%\titleformat{\section}{\bf\Large}{\arabic{section}}{24pt}{}
%\titleformat{\subsection}{\large}{\arabic{subsection}.}{12pt}{}
%\titlespacing*{\subsection}{0pt}{0pt}{1.5ex}

\parindent=0pt
\parskip=1em

\DeclarePairedDelimiter{\abs}{\lvert}{\rvert}
\DeclarePairedDelimiter{\norm}{\lVert}{\rVert}
\DeclarePairedDelimiter{\inpd}{\langle}{\rangle}
\DeclarePairedDelimiter{\ceil}{\lceil}{\rceil}
\DeclarePairedDelimiter{\floor}{\lfloor}{\rfloor}

\setminted{
  linenos=true, frame=lines, framesep=2mm,
  fontsize=\codesize
}
\definecolor{inlinebg}{rgb}{0.9, 0.9, 0.9}
\setmintedinline{
  bgcolor=inlinebg,
}

\renewcommand{\algorithmicrequire}{\textbf{Input:}}
\renewcommand{\algorithmicensure}{\textbf{Output:}}

\newcommand{\img}{\mathsf{i}}
\newcommand{\ex}{\mathsf{e}}
\newcommand{\dD}{\mathrm{d}}
\newcommand{\dI}{\,\mathrm{d}}

\newcommand{\ord}[1]{\opord\left(#1\right)}
\newcommand{\opord}{\operatorname{\mathcal{O}}}

\setenumerate{itemsep=0pt,topsep=0pt}

\title{數位電路實驗 Lab-3 教學手冊 -- 錄音機 \vspace{-0.2cm}}
\author{Team \#2 \\ B02901027 茅耀文, B02901178 江誠敏, B02901179 黃凱祺}
\begin{document}
\maketitle

\setcounter{section}{-1}
\section{前言}

\section{Chisel}
Chisel 是一個用來做高階硬體的語言,其本質上是一個 Scala 的 library 。
下面將會注重於比較用 Verilog 與 Chisel 編寫的差異和使用 Chisel 的好處。

\subsection{Basic Mappings from Verilog}
\hspace{2em}
\begin{minipage}{0.4\textwidth}
  \begin{minted}{verilog}
wire [n:0] foo;
reg [n:0] bar;
assign foo = a + b;
bar <= a + b;
  \end{minted}
\end{minipage}
\hspace{2em}
\begin{minipage}{0.4\textwidth}
  \begin{minted}{scala}
val foo = UInt(width=n)
val bar = Reg(UInt(width=n))
foo := a + b
bar := a + b
  \end{minted}
\end{minipage}
\subsection{Clock and Reset}
\begin{minted}{scala}
// Auto set bar to 0 at posedge of reset
// and change its value at posedge of clk
val bar = Reg(UInt(width=8), init=UInt(0))
when (condition) {
  // update register bar when "condition" is ture at posedge of clock
  bar := a | b
}
\end{minted}
對於每個 module 而言 Chisel 會自動提供一個 clock 和 reset 的 input ,
因此不需要像 Verilog 那樣自己寫。
當然如果有多個 clock 時也可以指定使用特定的 clock ,細節可參考
Multiple Clock Domains\footcite{chisel:multi-clock},在此不多做說明。
\subsection{Encapsulation}
隨著設計變得龐大,模組之間的 IO 規範會越來越多,在 Chisel 裡面可以用繼承
{\tt Bundle} 的方式來定義溝通界面,並在使用時再決定是輸入還是輸出。
與 Verilog 相比,封裝使 Chisel 的程式碼在閱讀和維護上都更容易。

\begin{minted}{scala}
// Encapsulation example
class SomeIO extends Bundle {
  val ready = Bool(false)
  val data = UInt(width=32)
  val control = UInt(width=8)
}

// Following code should be put in a Module or a Bundle
// Use SomeIO as input and output
val in = new SomeIO().asInput
val out = new SomeIO().asOutput
\end{minted}
\subsection{Object-Oriented}
除了封裝 IO 的格式之外,也可利用 Scala 語言本身的功能來實做物件導向的設計。
這是 Verilog 難以做到的。具體來說我們可以將一些接線的規則用函式包裝起來,
在需要的時候呼叫函式,如下面所示:

\begin{minted}{scala}
// Object-oriented design
class Button extends Bundle {
  val isPushed = Reg(Bool(false))
  when (isPushed) { isPushed := Bool(false) }
  def push = { isPushed := Bool(true) }
}

// Following code should be put in a Module or a Bundle
val button = new Button
when (condition) {
  button.push() // same as button.isPushed := Bool(true)
}
\end{minted}

需要注意的是,由於在使用 Module 的時候需要用 {\tt Module(new MyModule)}
的方式,故 Module 之間是不能呼叫函數的。 (也很合理)

\subsection{Testing with Ease}
Chisel 的一個重要功能是產生 C++ 的模擬程式來做電路測試。而測試相當易寫,
產生出來的測試檔效能也不錯。範列如下:
\begin{minted}{scala}
class TopTest(c: Top) extends Tester(c) {
  val ans = 10
  poke(c.io.start, 1) // set input wire "start" of Top to 1
  step(1) // go to the next clock cycle
  poke(c.io.start, 0)
  while (!peek(c.io.done)) {
    step(1)
    peek(c.io.out) // print the value of wire "out" of Top
  }
  expect(c.io.out, ans) // check if the value of "out" is equal to ans
}
\end{minted}

\subsection{Tips and Tricks}
Chisel 本身沒有定義型態的隱性轉換,這在編寫程式時相當不便。
比較好的方法是在一個檔裡寫下類似如下列的程式碼:
\begin{minted}{scala}
object Conv {
  implicit def intToUInt(x: Int): UInt = UInt(x)
  implicit def intToBoolean(x: Int): Boolean = if (x != 0) true else false
  implicit def booleanToInt(x: Boolean): Int = if (x) 1 else 0
  implicit def UIntToBool(x: UInt): Bool = x != 0
}
\end{minted}
並在其他的檔引入這些函數。

\subsection{Other Features}
Chisel 是建立在 Scala 之上,因此可以運用 Scala 的特性做 Parameterization 。
這也是 Verilog 所做不到的。
這邊的內容太多,在此不多做說明,細節可參考 Advanced Parameterization
\footcite{chisel:adv-param}。

除了上述內容之外, Chisel 還有提供相當多 Built-in 的功能可以使用,
相關資訊可參考 Chisel Cheatsheet\footcite{chisel:cs}。

\section{MyHDL}
不同於 Chisel , MyHDL 是建立在 Python 之上,主要用途是 High level modeling 。
轉換成 Verilog 的能力有限。其中可轉換成 Verilog 的部分除了 Python 自身 syntax
比較簡潔之外,並沒有比 Verilog 好寫,轉換出來的結果也跟直接寫 Verilog 差不太多,
整體來看並沒有太多的好處,因此這邊就不多說明。

\section{C-like Code to Verilog}
我們還額外實做了一個把一個我們自已定出的語言 Compile 成 Verilog code
的程式。

\subsection{Main algorithm}
C-like syntax AST -> Lexer/Parser -> State machine -> Verilog AST

\subsubsection{Statem machine construction}

\subsubsection{Async functions}

\subsection{example}
\inputminted{ver}{ref-codes/f.ver}

\inputminted{ver}{ref-codes/g.ver}

\subsection{Future works}
\subsubsection{State Compression}

\subsubsection{Modulize}


\end{document}

