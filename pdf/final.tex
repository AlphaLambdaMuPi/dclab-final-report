\documentclass[12pt, a4paper]{article}

\usepackage[hmargin=2.5cm, vmargin=2cm]{geometry}
\usepackage{amsthm, amssymb, mathtools, yhmath, graphicx}
\usepackage{fontspec}
\usepackage{type1cm, titlesec, titling, fancyhdr, tabularx}
\usepackage{color, float, hhline}

\usepackage{algorithmicx, algorithm, algpseudocode}
\usepackage[CheckSingle, CJKmath]{xeCJK}
\usepackage{CJKulem}
\usepackage{enumitem}
\setenumerate{itemsep=0pt}
\usepackage[abbreviations, binary-units]{siunitx}
\DeclareSIUnit\mybyte{Byte}
\usepackage[usenames]{xcolor}
\usepackage{tikz}
\usepackage{circuitikz}
\usepackage{hyperref}
\usepackage{minted}
\usepackage{lipsum}
\BeforeBeginEnvironment{minted}{\vspace*{-5mm}}
\AfterEndEnvironment{minted}{\vspace*{-5mm}}

\usepackage{titling}
\setlength{\droptitle}{-3em}

\usetikzlibrary{matrix}
\usetikzlibrary{decorations.markings}
\usetikzlibrary{decorations.pathmorphing}

\usepackage[backend=biber, style=verbose]{biblatex}
\bibliography{lab3-teach.bib}


\setmainfont{Linux Libertine O}
\setmonofont{Source Code Pro}
\setCJKmainfont{Source Han Sans TC}

\def\codesize{\fontsize{10}{15}\selectfont}
\def\normalsize{\fontsize{12}{18}\selectfont}
\def\large{\fontsize{14}{21}\selectfont}
\def\Large{\fontsize{16}{24}\selectfont}
\def\LARGE{\fontsize{18}{27}\selectfont}
\def\huge{\fontsize{20}{30}\selectfont}


%\titleformat{\section}{\bf\Large}{\arabic{section}}{24pt}{}
%\titleformat{\subsection}{\large}{\arabic{subsection}.}{12pt}{}
%\titlespacing*{\subsection}{0pt}{0pt}{1.5ex}

\parindent=0pt
\parskip=1em

\DeclarePairedDelimiter{\abs}{\lvert}{\rvert}
\DeclarePairedDelimiter{\norm}{\lVert}{\rVert}
\DeclarePairedDelimiter{\inpd}{\langle}{\rangle}
\DeclarePairedDelimiter{\ceil}{\lceil}{\rceil}
\DeclarePairedDelimiter{\floor}{\lfloor}{\rfloor}

\setminted{
  linenos=true, frame=lines, framesep=2mm,
  fontsize=\codesize
}
\definecolor{inlinebg}{rgb}{0.9, 0.9, 0.9}
\setmintedinline{
  bgcolor=inlinebg,
}

\renewcommand{\algorithmicrequire}{\textbf{Input:}}
\renewcommand{\algorithmicensure}{\textbf{Output:}}

\newcommand{\img}{\mathsf{i}}
\newcommand{\ex}{\mathsf{e}}
\newcommand{\dD}{\mathrm{d}}
\newcommand{\dI}{\,\mathrm{d}}

\newcommand{\ord}[1]{\opord\left(#1\right)}
\newcommand{\opord}{\operatorname{\mathcal{O}}}

\setenumerate{itemsep=0pt,topsep=0pt}

\title{數位電路實驗 Lab-3 教學手冊 -- 錄音機 \vspace{-0.2cm}}
\author{Team \#2 \\ B02901027 茅耀文, B02901178 江誠敏, B02901179 黃凱祺}
\begin{document}
\maketitle

\setcounter{section}{-1}
\section{前言}

\section{Chisel}

\subsection{Basic Mappings from Verilog}
{\tt wire [n:0] foo; } {\tt val foo = UInt(width=n) }

{\tt reg [n:0] bar; } {\tt val bar = Reg(UInt(width=n)) }

{\tt assign foo = a + b; } {\tt foo := a + b }

{\tt bar <= a + b; } {\tt bar := a + b }
\subsection{Clock and Reset}
\subsection{Encapsulation}
定義溝通界面

可在使用時再決定是 input 或是 output 。
\subsection{Object-Oriented}
物件導向的設計方式

注: 在使用 Module 的時候需要用 Module(new MyModule) 的方式,故
Module 之間是不能呼叫函數的。 (也很合理)

\subsection{Testing with ease}
Chisel 提供方便寫 Testbench 的方法

\subsection{Other Features}
由於 Chisel 是建立在 Scala 之上,因此可以運用 Scala 的特性做 Parameterization

另外 Chisel 也提供相當多 Built-in 的功能可以使用

\subsection{Tips and Tricks}
Chisel 本身沒有定義型態的隱性轉換,這在編寫程式時相當不便。
比較好的方法是在一個檔裡寫下:

並在其他的檔引入這些函數。

\section{MyHDL}
建立在 Python 之上,主要是用來做 High level modeling 。轉換成 Verilog
的能力有限。其中可轉換成 Verilog 的部分除了 Python 自身 syntax 比較簡潔之外,
並沒有比 Verilog 好寫,轉換出來的結果也就是跟直接寫 Verilog 差不太多,
因此這邊就不多說明。

\section{C-like Code to Verilog}
我們還額外實做了一個把一個我們自已定出的語言 Compile 成 Verilog code
的程式。

\subsection{Main algorithm}
C-like syntax AST -> Lexer/Parser -> State machine -> Verilog AST

\subsubsection{Statem machine construction}

\subsubsection{Async functions}

\subsection{example}
\inputminted{ver}{ref-codes/f.ver}

\inputminted{ver}{ref-codes/g.ver}

\subsection{Future works}
\subsubsection{State Compression}

\subsubsection{Modulize}


\end{document}

